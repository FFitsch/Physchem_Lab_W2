\documentclass[titlepage]{article}
\usepackage{fancyhdr}
\usepackage[margin=1.2in]{geometry}

\usepackage[utf8]{inputenc}
\usepackage[english]{babel}
\usepackage{csquotes}

\usepackage[backend=biber,style=apa,sorting=ynt]{biblatex}
\addbibresource{W2.bib}

\usepackage[breaklinks]{hyperref}

\usepackage{graphicx}
\usepackage{float}

\usepackage{derivative}

\usepackage{array,tabularx,calc}
\newlength{\conditionwd}
\newenvironment{conditions}[1][where:]{
        #1\tabularx{\linewidth-\widthof{#1}}[t]
        {>{$}l<{$} @{${}={}$} X@{}}}
  {\endtabularx\\[\belowdisplayskip]}


% Title Page -----------------------------------------------------------
\title{Protocol \\ W2 - Stirling Engine}
\author{Group F\\Jonas Adamer (12225913)\\Florian Fitsch (12218283)\\Leonhard Ritt (12208881)}
\date{Date of experiment: 2024/10/14\\Date of submission:}

\pagestyle{fancy}
\fancyhf{}
\fancyhead[R]{\thepage}
\fancyfoot[L]{W2 Stirlingmotor}
\fancyfoot[C]{Group F}
\fancyfoot[R]{Adamer, Fitsch, Ritt}

\begin{document}

% TITLE PAGE
\maketitle
\thispagestyle{empty}

% TABLE OF CONTENTS
\newpage
\tableofcontents
\thispagestyle{fancy}


\newpage
\section{Objective}
In this exercise, a Stirling engine is studied in order to determine some of its thermodynamic properties. Firstly, the engine is run manually as a heat pump, in order to assess the reversibility of the Stirling cycle through temperature measurements at the warm and cold sinks. Secondly, the engine is run by supplying heat through combustion of Ethanol as a fuel. Through measurements of temperature, pressure and volume, \(p/V\) diagrams are created and the following values are determined:
\begin{enumerate}
    \item Efficiency of heat transfer between heat source and engine.
    \item Mechanical efficiency of the engine.
    \item Total efficiency of the engine.
\end{enumerate}
Additionally, the engine's Frequencies at different temperatures are compared. 


\section{Theory}
The Stirling cycle is a thermodynamic cycle, in which a working fluid (in the case of this experiment air) undergoes four steps:
%
\begin{enumerate}
    \item An isothermal (no change in temperature) expansion of the air. Heat is taken up by the system from a hot reservoir, while work is given off to a flywheel.
    \item An isochoric (no change in volume) cooling of the air. No work is done by or put into the system.
    \item An isothermal contraction of the air. Heat is given off to a cold reservoir, and work is put into the system by the flywheel.
    \item An isochoric heating of the air. No work is done by or put into the system.
\end{enumerate}

\subsection{Total Work in a Stirling Cycle}
The first law of thermodynamics states, that the change of the internal energy of a system can be expressed as a sum of the heat which is added to or removed from the it, as well as the mechanical work, which is done by or put into it.
%
\begin{equation}
    \odif U = \fdif Q + \fdif W = \fdif Q - p \odif V
\end{equation}
\begin{conditions}
    U & Internal energy of the system [J] \\
    Q & Heat added to or removed from the system [J] \\
    W & Heat added to or done by the system [J] \\
    p & Pressure of the system [Pa] \\
    V & Volume of the system [m\textsuperscript{3}]
\end{conditions}
%
During the isothermal steps of the process, the internal energy \(U\) of the System does not change, meaning \(\fdif -Q = W = -p \odif V\). By integrating all terms and using the ideal gas law \(pV = nRT\), to substitute \(p\), the work which is carried out during an isothermal state change can be calculated:
%
\begin{equation} \label{eq_isothermal_work}
    -Q = W = -nRT \cdot ln \frac{V_2}{V_1}
\end{equation}
\begin{conditions}
    n & Amount of substance [mol] \\
    R & Ideal gas constant = 8.3145 [J mol\textsuperscript{-1} K\textsuperscript{-1}] \\
    T & Temperature of the system [K] \\
    V_2 & Volume at the end of isothermal state change [m\textsuperscript{3}] \\
    V_1 & Volume at the start of isothermal state change [m\textsuperscript{3}]
\end{conditions}
%
During the isochoric steps, \(\odif V\) is equal to zero, meaning no work is carried out. As such the change of internal energy is equal to the heat which is added to or removed from the system.
%
\begin{equation}
    \adif U = Q
\end{equation}
%
Since there is no change in Volume and thus no work during the isochoric steps, the total work which is carried out by the system during the Stirling cycle can be calculated by adding the work of both isothermal steps.
%
\begin{equation}
    W_{pV} = - nRT_1 \cdot \ln{\frac{V_2}{V_1}} - nRT_2 \cdot \ln{\frac{V_1}{V_2}} = nR(T2-T1) \cdot \ln{\frac{V_2}{V_1}}
\end{equation}
%
This work is equal to the enclosed area within the p/V diagram of the cycle.

\subsection{Efficiency of the Stirling Cycle}
It is known that the idealized Stirling cycle has a thermal efficiency equal to that of the idealized Carnot cycle.
%
\begin{equation}
    \eta = \frac{W_{pV}}{Q_1} = \frac{T_1 - T_2}{T_1}
\end{equation}
%
In a real-world Stirling engine however, this efficiency is not reached due to the overlap of the cycle's four steps with each other, along with other losses such as friction or during heat transfer. As such, other efficiency values are used to assess the quality of the real-world engine.

The first value of interest is the efficiency of heat transfer between the heat source (in this case a small lamp, fuelled by Ethanol) and the hot reservoir (bottom side) of the Stirling engine.
%
\begin{equation} \label{eq_eta_H}
    \eta_H = \frac{Q_1}{Q_H}
\end{equation}
\begin{conditions}
    \eta_H & Efficiency of heat transfer between heat source and hot reservoir \\
    Q_1 & Heat received at the hot reservoir [J] \\
    Q_H & Heat given off by the heat source [J]
\end{conditions}
%
Secondly, the mechanical efficiency, which relates the amount of usable work received at the flywheel to the work which is carried out by the system.
%
\begin{equation} \label{eq_eta_m}
    \eta_m = \frac{W_m}{W_{pV}}
\end{equation}
\begin{conditions}
    \eta_m & Mechanical efficiency of the engine \\
    W_m & Usable work received at the flywheel [J] \\
    W_{pV} & Work given off by the engine [J]
\end{conditions}
%
Thirdly, the total efficiency, which relates the amount of usable work received at the flywheel to the heat which is given off by the heat source.
%
\begin{equation}  \label{eq_eta}
    \eta = \frac{W_m}{Q_H}
\end{equation}
\begin{conditions}
    \eta & Total efficiency of the process
\end{conditions}
%
\newpage
The values used in these calculations can be determined as follows:
%
\begin{enumerate}
    \item \(Q_1\) is determined from Equation \ref{eq_isothermal_work} by applying the ideal gas law \(pV = nRT\):
    \begin{equation} \label{eq_Q1}
        Q_1 = p_1V_1 \cdot ln(\frac{V_2}{V_1})
    \end{equation}
    \begin{conditions}
        p_1 & Pressure at the start of isothermal expansion [Pa] \\
        V_1 & Volume at the start of isothermal expansion [m\textsuperscript{3}] \\
        V_2 & Volume at the end of isothermal expansion [m\textsuperscript{3}]
    \end{conditions}

    \item \(Q_H\) is determined by multiplying the molar heat of combustion of the fuel with the amount of fuel used, which in turn is calculated by dividing the mass of used fuel with its molar mass. The mass of used fuel can be determined by multiplying the fuel consumption rate of the lamp with the length of time over which the fuel was burned.
    \begin{equation} \label{eq_QH}
        Q_H = \Delta_c H_m \cdot \frac{m_{EtOH}}{M_{EtOH}} = Q_H = \Delta_c H_m \cdot \frac{k \cdot \adif t}{M_{EtOH}}
    \end{equation}
    \begin{conditions}
        \Delta_c H_m & Molar heat of combustion of burned fuel [J/mol] \\
        m_{EtOH} & Mass of burned fuel [g] \\
        M_{EtOH} & Molar mass of burned fuel [g/mol] \\
        k & Fuel consumption rate of the lamp [g/s] \\
        \adif t & Time period [s]
    \end{conditions}

    \item \(W_m\) is determined by calculating the work required to spin the flywheel up to its regular frequency from a standstill.
    \begin{equation} \label{eq_Wm}
        W_m = 4 \pi^2 m r^2 (\frac{f_f - f_0}{\adif t_{start}})
    \end{equation}
    \begin{conditions}
        m & Mass of flywheel [kg] \\
        r & Radius of flywheel [m] \\
        f_f & Frequency of the engine after several cycles [Hz] \\
        f_0 & Frequency of the engine during the first cycle [Hz] \\
        \adif t_{start} & Time between the start of the cycle used to determine \(f_0\) and end of the cycle used to determine \(f_f\) [s]
    \end{conditions}
    
    \item \(W_{pV}\) is determined by calculating the area enclosed in the p/V diagram.
\end{enumerate}


\newpage
\section{Experiment and Results}

\subsection{Reversibility of the Stirling Engine}
In order to assess the reversibility of the engine, a crank was attached to the flywheel, allowing for manual input of work into the system. The upper part of the engine (cold reservoir) was covered with 100 g of water, while the bottom part (hot reservoir) was left exposed to air.
The crank was then spun by hand, first clockwise (the direction in which it would run during operation as a regular heat engine), then counterclockwise, for one minute each. Special attention was paid towards trying to keep a constant angular velocity while cranking the flywheel, both during individual runs as well as from one run to another.
The temperatures on both sides of the engine were recorded before and after the runs, and are shown in Table \ref{tb_manual_temps}.\footnote{The inconsistent amount of decimal places is due to the lower accuracy of the bottom thermometer.}

\begin{table}[H]
    \centering
    \caption{Recorded temperatures (in °C) before and after spinning the flywheel}
    \label{tb_manual_temps}
    \begin{tabular}{|c|c|c|c|c|}
        \hline
        & \multicolumn{2}{|c|}{\textbf{Clockwise}} & \multicolumn{2}{|c|}{\textbf{Counter-Clockwise}}
        \\
        \hline
        & \textbf{Top} & \textbf{Bottom} & \textbf{Top} & \textbf{Bottom}
        \\
        \hline
        \(T_{Start}\) & 25.2 & 28 & 25.2 & 28
        \\
        \hline
        \(T_{End}\) & 25.2 & 27 & 25.0 & 29
        \\
        \hline
    \end{tabular}
\end{table}
%
\noindent These results show that, in it's usual configuration as a heat engine, heat is transferred from the bottom of the engine, where normally a heat source would be located, to the top of it. When run in reverse, the opposite phenomenon is observed, and heat is moved from the top of the engine to the bottom.

The change in heat on each side of the engine can be calculated using the specific heat capacities of the adjacent fluids, the mass of the fluid and the difference in temperature.
%
\begin{equation}
    Q = m \cdot c \cdot \adif T
\end{equation}
\begin{conditions}
    Q & Heat added to or removed to the system [J] \\
    m & Mass of the adjacent fluid [g] \\
    \adif T & Change in temperature [°C]
\end{conditions}
%
However, since the bottom side of the engine is exposed to the surrounding atmosphere, the corresponding mass is very large, and as such the change in temperature on the bottom end of the engine would be expected to be negligible. The temperature change which is observed however, is due to the heat being taken up or given off by the metal body of the engine itself.

\newpage
\subsection{Efficiency Values of the Stirling Engine}
In order to calculate the efficiency values of the engine, the crank was removed from the flywheel and replaced by a gear, which was connected to a pV-sensor via belt drive. The pV-sensor as well as the temperature-probes were in turn connected to a pVnT-measuring device. Temperature values at specific times in the experiment were obtained directly from the displays of this device, while pressure and volume data was measured over a continuous timespan using an oscilloscope connected to it. The engine frequency was determined at specific points of the experiment as the reciprocal value of the time of a single cycle, which was determined from the pressure and volume data.

In order to calculate \(Q_H\) as shown in Equation \ref{eq_QH}, the fuel consumption rate of the used lamp was determined by letting the flame burn for a specific duration and weighing the lamp before and afterwards.
%
\begin{equation}
    k = \frac{\adif m}{\adif t}
\end{equation}
\begin{conditions}
    k & Fuel consumption rate of the lamp [g/s] \\
    \adif m & Difference in mass of the lamp [g] \\
    \adif t & Time the lamp was lit for [s]
\end{conditions}
%
The experiment was conducted several times the recorded mass differences and durations as well as the resulting fuel consumption rates, their average, standard deviation and uncertainty are shown in Table \ref{tb_consumption_rate}.

\begin{table}[H]
    \centering
    \caption{Measured mass differences and burn times of the lamp and resulting fuel consumption ratios as well as their average, standard deviation and error margin}
    \label{tb_consumption_rate}
    \begin{tabular}{|c|c|c|}
        \hline
        \(\adif m\) [g] & \(\adif t\) [s] & k [g/s]
        \\
        \hline
        0.78 & 60 & 0.013
        \\
        \hline
        0.7 & 60 & 0.012
        \\
        \hline
        0.7 & 60 & 0.013
        \\
        \hline
        0.76 & 60 & 0.013
        \\
        \hline
        \hline
        & \(\overline{x}\) & 0.012
        \\
        \hline
        & \(s\) & 0.0006
        \\
        \hline
        & \(S\) & 0.0009
        \\
        \hline
    \end{tabular}
\end{table}

\noindent In the three\footnote{An additional fourth run was conducted at a temperature of 210 °C, for which data seems to have been incorrectly exported from the oscilloscope and which can thus not be evaluated.} runs of the main experiment - each at a different temperature - the lamp was put underneath the engine and ignited. Once the engine reached the desired temperature, the flywheel was put in motion by hand and the starting temperature was recorded. After the first few cycles, the recording of the oscilloscope was stopped and the data exported onto a USB-drive. Another recording was made and exported once the engine reached its full operating frequency and another temperature measurement was taken before the torch was extinguished and the engine came to a halt.

In order to convert the voltages measured by the oscilloscope into pressures and volumes, the following pre-determined conversions were used:
%
\begin{equation}
    V = \frac{U + 41}{56} \cdot 10^{-3}
\end{equation}
\begin{conditions}
    V & Volume [m\textsuperscript{3}] \\
    U & Measured current [V]
\end{conditions}
%
\begin{equation}
    p = (\frac{U + 1.05}{3.1} + p_0) \cdot 10^5
\end{equation}
\begin{conditions}
    p & Pressure [Pa] \\
    U & Measured current [V] \\
    p_0 & Ambient pressure [bar]
\end{conditions}
%
In Figure \ref{fig_start_pvt} the volume and pressure of the working fluid at different temperatures are shown plotted against time, while in Figure \ref{fig_constant_pvt_pv}, the same plots are shown at constant motion of the engine, in addition to the corresponding p/V-diagrams.

\begin{figure}[H]
    \centering
    \includegraphics[width=0.7\textwidth]{Figures/pvt\_start.png}
    \caption{Volume (shown in red) and pressure (shown in blue) of the working fluid, plotted against time, at the start of each experiment. The start and end of the first and fourth cycle, which are used to determine \(f_0\) and \(f_f\), were selected to coincide with the local maxima of the pressure graph, and are marked with black dots.}
    \label{fig_start_pvt}
\end{figure}

\begin{figure}[H]
    \centering
    \includegraphics[width=\textwidth]{Figures/pvt\_pv\_constant.png}
    \caption{Volumes (shown in red) and pressures (shown in blue) of the working fluid, plotted against time, at constant motion of the engine, are shown on the left for each temperature. The start and end of the first cycle is marked with black dots. The period in between these is used to plot the pressure against temperature, which is shown on the right. In addition, the minimum and maximum pressures during the cycle are similarly marked and used for later calculations.}
    \label{fig_constant_pvt_pv}
\end{figure}

\noindent Table \ref{tb_efficiency_runs} shows relevant data for the calculation of the Stirling engine's efficiency values. Values for \(Q_1\) are calculated using Equation \ref{eq_Q1}, with \(V_1\) and \(V_2\) taken as the maximum and minimum volumes during the marked cycle in Figure \ref{fig_constant_pvt_pv}. Since \(Q_1\) is calculated during isothermic expansion, the higher volume is used as \(V_2\). \(p_1\) is thus taken as the pressure at the time at which \(V_1\) is found. Values for \(Q_H\) are calculated using Equation \ref{eq_QH}, using a molar heat of combustion \(\Delta_c H_m\)(EtOH) of 1360 kJ/mol. Values for \(W_m\) are calculated using Equation \ref{eq_Wm}, using a mass of 4.7~kg and a radius of 0.17~m for the flywheel. \(W_pV\) is calculated as the enclosed area in the p/V diagram, using a supplied excel sheet.

Since \(Q_H\), unlike the other calculated values, is not determined for a single cycle, it needs to be normalized to the cycle duration as follows:
%
\begin{equation}
    Q_{H,corrected} = Q_{H,calculated} \cdot \frac{\adif t}{{\adif t}_{start}}
\end{equation}
\begin{conditions}
    Q_{H,corrected} & \(Q_H\) for once cycle [J] \\
    Q_{H,calculated} & \(Q_H\) as calculated using Equation \ref{eq_Q1} [J] \\
    \adif t & Duration of one cycle [s] \\
    {\adif t}_{start} & Time difference between start of first and end of fourth cycle, as used in Equation \ref{eq_Q1} [s]
\end{conditions}
%
\begin{table}[H]
    \centering
    \caption{}
    \label{tb_efficiency_runs}
    \begin{tabular}{|r||c|c|c|c|}
        \hline
        \(T_{start}\) [°C] & \textbf{190} & \textbf{230} & \textbf{250}
        \\
        \hline
        \hline
        \(f_0\) [Hz] & 0.93 & 1.10 & 1.23
        \\
        \hline
        \(f_f\) [Hz] & 1.48 & 1.79 & 1.95
        \\
        \hline
        \({\adif t}_{start}\) [s] & 3.33 & 2.81 & 2.53
        \\
        \hline
        \(f_{const}\) [Hz] & 3.41 & 4.35 & 3.36
        \\
        \hline
        \({\adif t}\) [s] & 0.29 & 0.23 & 0.30
        \\
        \hline
        \(V_1\) [m\textsuperscript{3}] & \(7.08 \cdot 10^{-4}\) & \(7.03 \cdot 10^{-4}\) & \(7.03 \cdot 10^{-4}\)
        \\
        \hline
        \(V_2\) [m\textsuperscript{3}] & \(7.97 \cdot 10^{-4}\) & \(7.93 \cdot 10^{-4}\) & \(7.93 \cdot 10^{-4}\)
        \\
        \hline
        \(p_1\) [Pa] & \(1.37 \cdot 10^5\) & \(1.37 \cdot 10^5\) & \(1.35 \cdot 10^5\)
        \\
        \hline
        \hline
        \(Q_1\) [J] & 11.48 & 11.62 & 11.45
        \\
        \hline
        \(Q_H\) [J] & 105.96 & 83.18 & 107.77
        \\
        \hline
        \(W_m\) [J] & 0.0391 & 0.0533 & 0.0831
        \\
        \hline
        \(W_{pV}\) [J] & 0.0506 & 0.1833 & 0.3247
        \\
        \hline
    \end{tabular}
\end{table}

Using Equations \ref{eq_eta_H}, \ref{eq_eta_m} and \ref{eq_eta}, the resulting efficiency values can then be caluclated:

\begin{table}[H]
    \centering
    \caption{}
    \label{tb_efficiency_values}
    \begin{tabular}{|r||c|c|c|c|}
        \hline
        \(T_{start}\) [°C] & \textbf{190} & \textbf{230} & \textbf{250}
        \\
        \hline
        \hline
        \(\eta_H\) & 0.11 & 0.14 & 0.11
        \\
        \hline
        \(\eta_m\) & 0.77 & 0.29 & 0.27
        \\
        \hline
        \(\eta\) & 0.00037 & 0.00064 & 0.00083
        \\
        \hline
    \end{tabular}
\end{table}

\newpage
\section{Discussion of Results and Uncertainty of Measurements}

% \printbibliography

\end{document}